\documentclass[submission,copyright,creativecommons]{eptcs}
\providecommand{\event}{SOS 2007} % Name of the event you are submitting to
\usepackage{breakurl}             % Not needed if you use pdflatex only.
\usepackage{underscore}           % Only needed if you use pdflatex.

\usepackage[utf8]{inputenc}
\usepackage[T1]{fontenc}

\title{A Bool in the Hand Is Worth Two in the Bush}
\author{
  James Wood
  \institute{Computer and Information Sciences\\
    University of Strathclyde\\
    Glasgow, United Kingdom}
  \email{james.wood.100@strath.ac.uk}
  \and
  Robert Atkey
  \institute{Computer and Information Sciences\\
    University of Strathclyde\\
    Glasgow, United Kingdom}
  \email{robert.atkey@strath.ac.uk}
}
\def\titlerunning{Bool in Hand, Two in Bush}
\def\authorrunning{J. Wood, R. Atkey}
\begin{document}
\maketitle

\begin{abstract}
A common practice in dependently typed programming is
\emph{correct-by-construction} programming -- where we write programs that carry statically a lot of the information that is needed to verify them.
For example, instead of having \texttt{filter} take a list of \texttt{xs} and
produce a list of \texttt{xs}, we declare a property \texttt{P} and return, as
well as the list of \texttt{xs}, a proof for each \texttt{x} in the returned
list that it satisfies \texttt{P}.
However, when we do correct-by-construction programming with fancy types, we
sometimes lose sight of how these functions compute.
In this paper, we rectify a long-standing bug in the Agda and Idris standard
libraries in their definitions of \texttt{Dec} -- the type family that captures
the decidability of a proposition -- so as to get back good computational
properties.
\end{abstract}

\section{Introduction}

% Where intrinsic verification meets extrinsic verification
% Making intrinsic verification amenable to extrinsic verification

In dependently typed programming, verification of programs may be done in two
different styles: \emph{intrinsic} and \emph{extrinsic}.
When verifying a program \emph{intrinsically}, that program will be written with an expressive type, expressing not only data representation, but also
many of the program's correctness conditions.
Correspondingly, the program itself will specify not only the intended
operational behaviour, but also a proof that it fits the specification.
On the other hand, when using the \emph{extrinsic} style, the program will
have a relatively simple type, and supplementary definitions (lemmas) are
needed to establish the correctness of the program.
Both of these styles have advantages and disadvantages, and a typical program
will be verified in a mixture of the two styles.

With intrinsic and extrinsic styles forming a spectrum, we will occasionally
find that two parts of a larger program have a mismatch in styles.
This will particularly be the case when a reusable library is written in one
style, and we want to write part of an application differently.
In this paper, we spot some instances where this mismatch, combined with a
common approach to intrinsic verification, cause unnecessary difficulty for the
users of a library.
We then proceed to suggest an approach to intrinsic verification which does not
have these issues.

We argue that our new approach to intrinsic verification is entirely
backwards-compatible with the old approach.
This claim is supported by the application of our technique to the definition
and properties of \texttt{Dec} in Agda's standard library.
These changes were made incrementally, in such a way that each change was an
improvement, and the entire library continued to type-check after each change.

\section{Problem}

We are writing a library of vectors and matrices.
We are going to do this in the style of a typical early undergraduate course,
where we allow ourselves some dependent types for matrix dimensions, but are
otherwise going to take an extrinsic verification approach.
In particular, we want to define the identity matrix and matrix multiplication,
and prove that these obey the usual algebraic laws.

We will define matrices as functions from a pair of indices to a coefficient.
In Agda, this means \texttt{Fin m * Fin n -> Carrier}, where \texttt{Carrier} is
the type of elements of the semiring we are working in.

In this setting, the $m \times m$ identity matrix is the function mapping equal
indices to $1$ and unequal indices to $0$.
To implement this, we look to the standard library to find a decision procedure
for two \texttt{Fin m} elements.
It behaves as if it were defined as follows.

\dots dec eq on Fin

We use this to implement the identity matrix.

\dots first implementation of identity matrix

When proving lemmas about matrix operations, we often proceed by induction on
one of the indices of the result.
When other indices of the same dimension appear, we are usually made to consider
all of the cases where each index is either \texttt{zero} or \texttt{suc _}.
In particular, we inspect all of the cases of the identity matrix:
\texttt{1M zero zero}, \texttt{1M zero (suc j)}, \texttt{1M (suc i) zero}, and
\texttt{1M (suc i) (suc j)}.
We find, conveniently, that the first three compute, yielding $1$, $0$, and $0$, respectively.
However, the \texttt{suc}-\texttt{suc} case is stuck.
We have to manually deploy equality lemmas to show that this is
\texttt{1M i j}.

Note that if we had defined \texttt{1M} directly as follows, the
\texttt{suc}-\texttt{suc} case would compute, as suggested by the final equation
of the definition.

\dots direct implementation of identity matrix

\section{Solution}

The \texttt{1M (suc i) (suc j)} case fails to compute because it is stuck on the
decision \texttt{i =? j}.
This is needed to compute \texttt{suc i =? suc j}.
However, what we actually need for the identity matrix is only whether the
decision is \texttt{yes} or \texttt{no}, and not the proofs that these were the
correct decision.
This is what \texttt{floor} leaves us with.
What we can see by inspection of the decision procedure is that
\texttt{suc i =? suc j} is going to be \texttt{yes _} exactly when
\texttt{i =? j} is \texttt{yes _}.
If we could observe this as a definitional equality, we would also get
\texttt{1M (suc i) (suc j)} to be definitionally equal to \texttt{1M i j}, as we wanted.

Our solution is to make explicit the invariant that a \texttt{yes _} is mapped
to another \texttt{yes _} and a \texttt{no _} is mapped to another
\texttt{no _}.
To do this, we start with a boolean value, encoding whether the decision result
is a \texttt{yes _} or a \texttt{no _}.
We then pair this with a proof that this decision was correct.
Crucially, because the computation of the first half of a pair is independent of
the second half, it is possible to express in a way visible to definitional equality that being a \texttt{yes _} is conserved.

\nocite{*}
\bibliographystyle{eptcs}
\bibliography{generic}
\end{document}
